\documentclass[12pt,a4j]{ltjarticle}
\evensidemargin -0.3in
\oddsidemargin -0.3in
\topmargin -1.0in
\textheight 10.2in
\textwidth 6.9in
\usepackage{color}
\usepackage{amsmath}
\newcommand{\bm}[1]{{\mbox{\boldmath $#1$}}}

\title{一般化されたオームの法則}
\author{堀田英之}

\begin{document}
\maketitle
    \S 2.1.3に登場する, 一般化されたオームの法則の導出を示す. かなりの非自明な近似を使っており, 導出は極めて困難.    
    ここでは教科書に沿って, 一価のプラズマについて考える. よって$n_i=n_e$.
    電子, 水素イオン(陽子)の運動方程式をそれぞれ書き下すと, それぞれの粒子の電荷量を$-e$, $e$とする.
    \begin{align}
        \frac{\partial}{\partial t}\left(n_em_e\bm{v}_e\right) 
        + \nabla\cdot\left(n_em_e\bm{v}_e\bm{v}_e\right) =&
         - \nabla p_e -
        en_e\left(\bm{E}+\bm{v}_e\times\bm{B}\right) + \bm{R}_{ei} + \bm{R}_{en} 
        \label{eom_electron}
        \\
        \frac{\partial}{\partial t}\left(n_im_i\bm{v}_i\right) 
        + \nabla\cdot\left(n_im_i\bm{v}_i\bm{v}_i\right) =&
         - \nabla p_i + 
        en_i\left(\bm{E}+\bm{v}_i\times\bm{B}\right) + \bm{R}_{ie} + \bm{R}_{in}
        \label{eom_proton}
    \end{align}
    それぞれの粒子に対する連続の式
    \begin{align}
        \frac{\partial n_*}{\partial t} = -\nabla\cdot\left(n_*\bm{v}_*\right)
    \end{align}
    が成り立っているので, 
    \begin{align}
        n_*m_*
        \left[
        \frac{\partial \bm{v}_*}{\partial t} + \left(\bm{v}_*\cdot\nabla\right)\bm{v}_*
        \right] = \frac{\partial}{\partial t}\left(n_*m_*\bm{v}_*\right) 
        + \nabla\cdot\left(n_*m_*\bm{v}_*\bm{v}_*\right)
    \end{align}
    という関係が成り立っていることに注意.
    ここで、一流体近似で用いられる速度$\bm{v}$, 電流密度$\bm{j}$を以下のように定義する
    \begin{align}
        \bm{v} =& \frac{n_em_e\bm{v}_e + n_im_i\bm{v}_i}{n_em_e + n_im_i} \sim
        \frac{m_e\bm{v}_e + m_i\bm{v}_i}{m_e +m_i}\\
        \bm{j} =& e\left(-n_e\bm{v}_e + n_i\bm{v}_i\right) \sim en_e \left(-\bm{v}_e + \bm{v}_i\right)
    \end{align}
    準中性条件から$n_e\sim n_i$を用いた.
    また, $m_e<<m_i$を用いて以下の近似をおこなう.
    \begin{align}
        \bm{v} =& \frac{m_i}{m_e + m_i}\left(\frac{m_e}{m_i}\bm{v}_e + \bm{v}_i\right)\sim \bm{v}_i \label{total_velocity}\\
        \bm{v}_e \sim& \bm{v}_i - \frac{\bm{j}}{en_e} \sim \bm{v} - \frac{\bm{j}}{en_e} \label{electron_velocity}
    \end{align}
    となる.
    式(\ref{eom_proton})$\times m_e/m_i-$式(\ref{eom_electron})を実行すると
    \begin{align}
        \frac{m_e}{e}
        \frac{\partial}{\partial t}
        \left[
        en_e\left(
        -\bm{v}_e + \bm{v}_i
        \right)
        \right] +& 
        \frac{m_e}{e}
        \nabla\cdot
        \left[
            en_e
            \left(
            -\bm{v}_e\bm{v}_e + \bm{v}_i \bm{v}_i
            \right)
        \right] \nonumber   \\ =&
        \nabla p_e 
        -\frac{m_e}{m_i}\nabla p_i  +
        en_e
        \left(\bm{E}+\bm{v}_e\times\bm{B}\right)
        + \frac{m_e}{m_i}en_i
        \left(\bm{E}+\bm{v}_i\times\bm{B}\right) \nonumber\\
        &+ \bm{R}_{ei} + \bm{R}_{en} + 
        \frac{m_e}{m_i}\left(\bm{R}_{ie} + \bm{R}_{in}\right)
    \end{align}
    左辺第二項の$\nabla\cdot$の中身を式(\ref{total_velocity})と(\ref{electron_velocity})を用いて変形する.
    \begin{align}
        en_e\left(-\bm{v}_e\bm{v}_e + \bm{v}_i\bm{v}_i\right) =& en_e
        \left[
            -\left(\bm{v}-\frac{\bm{j}}{en_e}\right)\left(\bm{v}-\frac{\bm{j}}{en_e}\right)
            + \bm{v}\bm{v}
        \right] \nonumber\\
        =& \bm{v}\bm{j} + \bm{j}\bm{v} + \frac{\bm{jj}}{en_e}
    \end{align}
    最後の項はPriestの教科書では無視されているように見える. 無視する妥当性はよくわからない. 右辺第三項は, 式(\ref{electron_velocity})を代入して, 
    \begin{align}
        en_e\left(\bm{E}+\bm{v}_e\times\bm{B}\right) = en_e
        \left(
            \bm{E} + \bm{v}\times\bm{B} - \frac{\bm{j}}{en_e}\times\bm{B}
        \right)
    \end{align}
    電子と水素イオンの衝突項は, 一般に以下のように表される.
    \begin{align}
        \bm{R}_{ei} =& \frac{n_em_e\left(\bm{v}_i-\bm{v}_e\right)}{\tau_{ei}} \\
        \bm{R}_{ie} =& \frac{n_im_i\left(\bm{v}_e-\bm{v}_i\right)}{\tau_{ie}}
    \end{align}
    作用反作用を考えると$\bm{R}_{ei}=\bm{R}_{ie}$なので, $m_e/m_i\bm{R}_{ie}$の項は無視できることがわかる.
    ここで衝突項は以下のように変形できる.
    \begin{align}
        \bm{R}_{ei} = \frac{m_e}{e\tau_{ei}}\bm{j} = \frac{m_e}{eB}\frac{1}{\tau_{ei}}B\bm{j} = 
        \frac{B\bm{j}}{\Omega_{e}\tau_{ei}}
    \end{align}
    ここで, $\Omega_e=eB/m_e$は電子のジャイロ振動数(磁場$B$の下で円運動するときの振動数)である.
    電子と中性粒子の衝突項は, $\bm{v}_n\sim\bm{v}_i$が仮定されていると思われる. そうすると電子とイオンの衝突項と同様に
    \begin{align}
        \bm{R}_{en} = \frac{B\bm{j}}{\Omega_e \tau_{en}}
    \end{align}
    と書くことができる.
    \\
    ここまでをまとめると,
    \begin{align}
        \frac{m_e}{e}
        \left[
        \frac{\partial\bm{j}}{\partial t} + \nabla\cdot\left(\bm{vj}+\bm{jv} + \frac{\bm{jj}}{en_e}\right)
        \right] = en_e\left(\bm{E}+\bm{v}\times\bm{B} + \frac{\nabla p_e}{en_e}\right) - \bm{j}\times\bm{B}
        +
        \left(
        \frac{1}{\Omega_e\tau_{ei}} + \frac{1}{\Omega_e\tau_{en}}
        \right)B\bm{j}
    \end{align}
    となる.\\
    最後の項はプラズマと中性粒子の相互作用によるもの. ここからは, 非常に
    一流体近似したときの運動方程式を書き下すと以下のようになる.
    \begin{align}
        \rho\left[
        \frac{\partial \bm{v}}{\partial t} + 
        \left(
            \bm{v}\cdot\nabla
        \right)\bm{v}
        \right] = -\nabla p + \bm{j}\times\bm{B} + \bm{R}_{in}
    \end{align}
    ここで一流体となった場合でも, 運動量のほとんどは水素イオンが担うので, 衝突項はイオンと中性粒子のものを考えた.
    ここでは, 圧力勾配力, ローレンツ力, 衝突項が釣り合っていると考える.
    衝突項をどのようにすればいいかよくわからないが, Priestの教科書では以下のように定義しているように見える。これはスタンダードな方法ではないだろう.
    \begin{align}
        \bm{R}_{in} = -\frac{\left(n_a+n_e\right)^2}{n_a^2e}\frac{m_e}{\tau_{in}}
        \left(\bm{v}_n - \bm{v}_i\right)
    \end{align}
    誘導電場は$\bm{v}\times\bm{B}$であるが, 中性粒子によるドリフトを考えると
    誘導電場は$\bm{v}_i\sim\bm{v}$を考慮すると
    \begin{align}
        \bm{v}\times\bm{B} = \bm{v}_n\times\bm{B} + \frac{n_a^2}{\left(n_a + n_e\right)^2}\frac{\tau_{in} e}{m_e}\left[\nabla p\times\bm{B} - \left(\bm{j}\times\bm{B}\right)\times\bm{B}\right],
    \end{align}
    となり, 誘導電場に最後の項を足すのが良いことがわかる. 教科書の一般化されたオームの法則が証明可能になる(中性粒子との衝突項はかなり怪しいが). 一般的な教科書では, 違う導き方をしているよう.

\end{document}